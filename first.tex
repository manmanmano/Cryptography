\documentclass[12pt]{article}
\usepackage{amsmath, booktabs, caption, scrextend}


\title{First Assignment}
\author{Mariano D'Angelo}
\date{$1^{st}$ March 2022}

\begin{document}

\maketitle


\section*{Task 1}
Calculate GCD(a, b) and find Bezout's identity for a=2022, b=752. \\\\
\begin{tabular}{ | c | c | c | } \hline
    \textbf{rem} & \textbf{val} & \textbf{expr} \\ \hline
    $r_0$ & 2022 & a \\
    $r_1$ & 752 & b \\
    $r_2$ = $r_0$ mod $r_1$ & 518 & a - 2b \\
    $r_3$ = $r_1$ mod $r_2$ & 234 & b - (a - 2b) = 3b - a \\
    $r_4$ = $r_2$ mod $r_3$ & 50 & (a - 2b) - 2(3b - a) = 3a - 8b \\
    $r_5$ = $r_3$ mod $r_4$ & 34 & (3b - a) - 4(3a - 8b) = 35b - 13a \\
    $r_6$ = $r_4$ mod $r_5$ & 16 & (3a - 8b) - (35b - 13a) = 16a - 43b \\
    $r_7$ = $r_5$ mod $r_6$ & 2 & (35b - 13a) - 2(16a - 43b) = 121b - 45a \\ \hline
    $r_8$ = $r_6$ mod $r_7$ & 0 & \\ \hline
\end{tabular}

\noindent \newline \newline The gcd of 2022 and 752 is \textbf{2}. \newline
From the final line of the table we can see that Bezout's identity is fulfilled
with \textbf{-45} for x and \textbf{121} for y.\newline
This can be checked with: $(2022 \cdot (-45)) + (752 \cdot 121)$ = 2

\pagebreak


\section*{Task 2} 

Solve the following congruences: \\

\noindent 1) x + 17 = 23 (mod 37) \textbar{} -17
\begin{addmargin}[1.22em]{0em}
x + 17 - 17 = 23 - 17 (mod 37) \\
x = 6 (mod 37) \\
x = \textbf{6} \\
\end{addmargin}

\noindent2) x + 42 = 19 (mod 51) \textbar{} -42
\begin{addmargin}[1.22em]{0em}
x + 42 - 42 = 19 - 42 (mod 51) \\
x = -23 (mod 51) \\
x = -23 + 51 (mod 51) \\   
x = 28 (mod 51) \\
x = \textbf{28}
\end{addmargin}


\section*{Task 3} 

Solve the following congruences: \\

\noindent 1) $23^{37}$ mod 40 = 
\begin{addmargin}[1.22em]{0em}
$23 \cdot 23^{36}$ mod 40 = \\
$23 \cdot 23^{12\cdot3}$ mod 40 = \\
$23 \cdot 23^{4\cdot3\cdot3}$ mod 40 = \\
$23 \cdot 23^{2\cdot2\cdot3\cdot3}$ mod 40 = \\
$23 \cdot 529^{2\cdot3\cdot3}$ mod 40 = \\
$23 \cdot (529^{2\cdot3\cdot3}$ mod 40) mod 40 = \\
$23 \cdot 9^{2\cdot3\cdot3}$ mod 40 = \\
$23 \cdot 81^9$ mod 40 = \\
$23 \cdot 1^9$ mod 40 = 
\textbf{23}
\end{addmargin}

\pagebreak

\noindent 2) $(-133)^{100}$ mod 10 = 
\begin{addmargin}[1.22em]{0em}
(-133 mod 10)$^{100}$ mod 10 = \\
$7^{100}$ mod 10 = \\
$7^{4 \cdot 25}$ mod 10 = \\
$2401^{25}$ mod 10 = \\
$1^{25}$ mod 10 = 
\textbf{1}
\end{addmargin}

\section*{Task 4}


\section*{Task 5}

Assume that the Affine cipher is implemented $Z_{89}$, not in $Z_{26}$.
\newline

\noindent
\textbf{1. Write down encryption and decryption functions for this modification of Affine cipher.}\newline

\noindent a, b - together form the key k(a, b) \\
m - plaintext message \\
c - ciphertext \\\\
\noindent The encryption key is: $E_m$ = am + b mod 89 \\
The decryption key is: $D_c$ = $a^{-1} \cdot (c - b)$ mod 89 \\

\noindent \textbf{2. What is the number of possible keys?}
\newline
\noindent
\\The number of possible keys is: $89 \cdot \Theta(89)$ or $89 \cdot 88$. 88 is 
what we get from the Euler's function, following the property: $\Theta(p) = p - 1$, 
where p represents a prime number. This means that there is a total of 7832 possible
keys.
\newline

\noindent \textbf{3. Suppose that modulus p = 89 is public. Malicious Eve intercepts 
two ciphertexts encrypted with the same key sent from Alice to Bob $c_1$ = 1 and
$c_2$ = 69. Assume that Eve also managed to find out that the corresponding 
plaintexts are $m_1$ = 10 and $m_2$ = 7. Find out the encryption key and use it 
to encrypt message $m_3$ = 13.}
\newline

\noindent Firstly, we write down the following system of equations:
$$\begin{cases}
    69 = 7a + b \text{mod 89} \\
    1 = 10a + b \text{mod 89} 
\end{cases}$$

Now we can solve it and find the key k(a, b). 
$$\begin{cases}
    69 = 7a + b \text{mod 89} \hspace{0.1cm} | \cdot (-1) \\
    1 = 10a + b \text{mod 89} 
\end{cases}$$

$$\begin{cases}
    -69 = -7a - b \text{mod 89} \\
    1 = 10a + b \text{mod 89} 
\end{cases}$$

\noindent \\After this we get: \\

\noindent -69 + 1 = -7a + 10a + 0 mod 89 \\
-68 = 3a mod 89 \\
(-68 mod 89) = 3a mod 89 \\
21 = 3a mod 89 \textbar{} : 3\\
7 = a mod 89 \\
a = 7 mod 89 \\
a = 7 \\

\noindent Now that we have a we can find b by just replacing a's value into one of the equations in the system. \\

\noindent $69 = 7 \cdot 7$ + b mod 89 \\
69 = 49 + b mod 89 \\
69 - 49 = b mod 89 \\
20 = b mod 89 \\
b = 20 mod 89 \\
b = 20 \\

\noindent a and b form the encryption key \textbf{k(7, 20)}.
Knowing that $m_3$ = 13 we can insert it into the Affine encryption function
$E_m$ = am + b mod 89, knowing that a = 7 and b = 20. \\
\pagebreak

\noindent The ciphertext we get is: \\

\noindent $7 \cdot 13$ + 20 mod 89 = \\
91 + 20 mod 89 = \\
111 mod 89 = 22 \\

\noindent Ciphertext $c_3$ is \textbf{22}.

\section*{Task 6}


\end{document}
