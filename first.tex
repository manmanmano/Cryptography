\documentclass[12pt]{article}
\usepackage{amsmath, booktabs, caption, scrextend}


\title{First Assignment}
\author{Mariano D'Angelo}
\date{February 2022}

\begin{document}

\maketitle

\noindent \textbf{Task 1}
\begin{equation}
\begin{tabular}{ | c | c | c | } \hline
    \textbf{rem} & \textbf{val} & \textbf{expr} \\ \hline
    $r_0$ & 2022 & a \\
    $r_1$ & 752 & b \\
    $r_2$ = $r_0$ mod $r_1$ & 518 & a - 2b \\
    $r_3$ = $r_1$ mod $r_2$ & 234 & b - (a - 2b) = 3b - a \\
    $r_4$ = $r_2$ mod $r_3$ & 50 & (a - 2b) - 2(3b - a) = 3a - 8b \\
    $r_5$ = $r_3$ mod $r_4$ & 34 & (3b - a) - 4(3a - 8b) = 35b - 13a \\
    $r_6$ = $r_4$ mod $r_5$ & 16 & (3a - 8b) - (35b - 13a) = 16a - 43b \\
    $r_7$ = $r_5$ mod $r_6$ & 2 & (35b - 13a) - 2(16a - 43b) = 121b - 45a \\ \hline
    $r_8$ = $r_6$ mod $r_7$ & 0 & \\ \hline
\end{tabular}
\end{equation}
\\\\
\textbf{Task 2} \\
1) x + 17 = 23 (mod 37) \textbar{} -17
\begin{addmargin}[1.2em]{0em}
   x + 17 - 17 = 23 - 17 (mod 37) \\
   x = 6 (mod 37) \\
   x = 6\\
\end{addmargin}
\noindent
2) x + 42 = 19 (mod 51) \textbar{} -42 
\begin{addmargin}[1.2em]{0em}
   x + 42 - 42 = 19 - 42 (mod 51) \\
   x = -23 (mod 51) \\
   x = -23 + 51 (mod 51) \\   
   x = 28 (mod 51) \\
   x = 28 \\
\end{addmargin}

\textbf{Task 3} \\
\\\\

\textbf{Task 4}

\end{document}
