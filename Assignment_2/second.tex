\documentclass{article}
\usepackage{booktabs, amsmath, amsfonts}

\title{Second Assignment}
\author{Mariano D'Angelo}
\date{4th April 2022}

\begin{document}

\maketitle


\section*{Task 1 | Big O Notation}
Prove that $O(n^2)= an^2 + bn - c$, where \textit{a} - is first 2 digits of your student
code, \textit{b} - is 3rd and 4th digit of your student code, \textit{c} - last 
two digits of your student code. \\\\
Code 201752IVSB, then \textit{a} = 20, \textit{b} = 17, \textit{c} = 52, \\
equation is $O(n^2) = 20n^2 + 17n - 52$ \\\\
From theory we know that $f(n) \le c \cdot g(n)$, so we get: \\\\
$20n^2 + 17n - 52 \le c \cdot n^2$ \\\\
To simplify our equation we can choose that $c = 21$. Now let's solve this:\\\\
$20n^2 + 17n - 52 \le 21n^2$ \\
$20n^2 - 21n^2 + 17n - 52 \le 0$ \\
$- n^2 + 17n - 52 \le 0$ \textbar{} $\cdot -1$\\
$n^2 - 17n + 52 \ge 0$ \\\\
By taking the derivative of the quadratic formula we get: \\\\
$2n - 17 \ge 0$ \\
$2n \ge 17$ \\
$n = 8.5$ \\\\
Meaning that the function starts to grow again after the value 8.5 is encountered. \\\\
By solving this quadratic formula equation we get the following values: \\\\
$n_1 = 13, n_2 = 4$, so $n < 4$ V $n > 13$ \\\\
Meaning that the function is positive only when bigger than 4 or 13. \\\\
As a result we get | $c = 21, n = 14$


\section*{Task 2 | Complexity theory}
Give an example of a \textit{search} problem and corresponding \textit{decision} problem,
which was not discussed in the lectures. \\\\
Is $x \in \mathbb{Z}$ positive or negative? \\\\
Consider the corresponding verification function $V(x,y)$:
$$\begin{cases}
    1 \text{  where } x = y + z \text{ and } (y \ge 0 \text{ and }z \ge 0) \text{ or } (+y > -|z| \text{ and viceversa}) \\
    0 \text{  if otherwise}
\end{cases}$$


\end{document}
