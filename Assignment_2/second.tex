\documentclass{article}
\usepackage{booktabs, amsmath, amsfonts}

\title{Second Assignment}
\author{Mariano D'Angelo}
\date{4th April 2022}

\begin{document}

\maketitle


\section*{Task 1 | Big O Notation}
Prove that $O(n^2)= an^2 + bn - c$, where \textit{a} - is first 2 digits of your student
code, \textit{b} - is 3rd and 4th digit of your student code, \textit{c} - last 
two digits of your student code. \\\\
Code 201752IVSB, then \textit{a} = 20, \textit{b} = 17, \textit{c} = 52, \\
equation is $O(n^2) = 20n^2 + 17n - 52$ \\\\
From theory we know that $f(n) \le c \cdot g(n)$, so we get: \\\\
$20n^2 + 17n - 52 \le c \cdot n^2$ \\\\
To simplify our equation we can choose that $c = 21$. Now let's solve this:\\\\
$20n^2 + 17n - 52 \le 21n^2$ \\
$20n^2 - 21n^2 + 17n - 52 \le 0$ \\
$- n^2 + 17n - 52 \le 0$ \textbar{} $\cdot -1$\\
$n^2 - 17n + 52 \ge 0$ \\\\
By taking the derivative of the quadratic formula we get: \\\\
$2n - 17 \ge 0$ \\
$2n \ge 17$ \\
$n = 8.5$ \\\\
Meaning that the function starts to grow again after the value 8.5 is encountered. \\\\
By solving this quadratic formula equation we get the following values: \\\\
$n_1 = 13, n_2 = 4$, so $n < 4$ V $n > 13$ \\\\
Meaning that the function is positive only when bigger than 4 or 13. \\\\
As a result we get | $c = 21, n = 14$


\section*{Task 2 | Complexity theory}
Give an example of a \textit{search} problem and corresponding \textit{decision} problem,
which was not discussed in the lectures. \\\\
Is $x \in \mathbb{Z}$ positive or negative? \\\\
Consider the corresponding verification function $V(x,y)$:
$$\begin{cases}
    1 \text{  where } x = y + z \text{ and } (y \ge 0 \text{ and }z \ge 0) \text{ or } (+y > -|z| \text{ and viceversa}) \\
    0 \text{  if otherwise}
\end{cases}$$
Search Problem (Summing): Given a positive \textit{x}, find \textit{y} such that
$V(x,y) = 1$. \\\\
Decision Problem (Positiveness/Negativeness): Given \textit{x}, decide if there is 
\textit{y} such that $V(x,y) = 1$.


\section*{Task 3 | Block ciphers}
Assume you are the agent of Mission Impossible. Top-level agents have decided that your
agency will use AES-128 block cipher for its missions. You are given a task to choose a 
suitable encryption mode for the following mission scenarios:
\begin{enumerate}
    \item Encryption of the agent's 6 digit identification number stored in the 
        Super Secret Database (SSD)
    \item Encryption of a document that will be sent via email.
\end{enumerate}
Please, motivate your answer.
\begin{enumerate}
    \item In order to encrypt a 6 digit identification number I would use ECB. Even though
        it is not that safe when it comes to the encryption of multiple blocks (lack of diffusion), 
        it can do well when it comes to the encryption of a single block. If we think of the 6 digit
        number as an integer it only occupies 4 bytes and we are encrypting with an 128 bit
        AES. Besides the lack of diffusion it has many positive sides, such as parallelizable
        encryption and decryption, and the possibility of random read.
    \item In order to encrypt a document I would use CTR. This mode of encryption has 
        many benefits, both encryption and decryption are parallelizable, random read is possible,
        faulty blocks affect only their current block.
        CTR uses a counter that is encrypted with the cipher text and increases 
        with each block, so repeating ciphertext will not appear (good diffusion). No padding is used 
        in the CTR mode (ciphertext is the same length as the plaintext) avoiding the padding oracle attack.
\end{enumerate}


\section*{Task 4 | Modes of encryption}
Alice wants to send a message \textit{m}, encrypted with block cipher, to Bob. The message m
is split into 4 blocks of equal length $m = m_0||m_1||m_2||m_3$ and encrypted using 3DES.
However, a transmission error occurs (or malicious Carol got access to the channel) and
one bit of the ciphertext $c_1$ changes its value. \\
Bob receives the ciphertexts, decrypts them and gets the following message 
$m' = m_0'||m_1'||m_2'||m_3'$. 
Please, explain to Bob how many bits (approximately) are expected to
be wrong in each block $m_i'$ if Alice used ECB and CBC modes. \\\\
3DES encrypts in blocks of 64 bits. So let's suppose each block contains about 60 bits of information. \\
If Alice used ECB, the whole block in which the error resuted ($m_1$), would
become faulty, so we would lose approximately 60 bits of data. ECB divides the plaintext
into blocks and every block is encrypted with the same key and algorithm. Because no parts
of the faulty block are used in the other blocks, the error is isolated. \\
If Alice used CBC, the error would be much worse than the one for ECB. This is because the
resulting ciphertext of the previous block is used to encrypt the current block and so on. If
we get the error in $m_1$ there will be two faulty blocks, the current one and the previous one (the text is being decrypted).
So we will have approximately 120 bits that are expected to be wrong.




\section*{Task 5 | Diffie-Hellman}
Consider the Diffie-Hellman protocol with $\alpha$ = 3 and \textit{p} = 673. Alice chooses
\textit{A} = 95 and Bob chooses \textit{B} = 240. Compute the messages that Alice and Bob
send to each other and the final shared key.


\section*{Task 6 | Group theory}
Do the following sets with defined operations form a group? Provide explanation.
\begin{itemize}
    \item $\mathbb{Z}_{10}^+$ (\textit{Integers modulo 10 under addition})
    \item $\mathbb{Q}$ (\textit{Rational numbers under multiplication}) 
    \item $\mathbb{Z}_{37}^-$ (\textit{Intergers under subtractions})
    \item $(\mathbb{C} - 0, *)$ (\textit{Complex numbers under multiplication})
\end{itemize}


\end{document}
